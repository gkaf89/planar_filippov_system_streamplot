\documentclass[english, a4paper]{article}

\usepackage{StyleFile}
\usepackage{LayoutFile}

\newcommand{\makeDiagramPath}[1]{Resources/Diagrams/#1}
\tikzexternalize[prefix=tikz-gen/] % Note: doesn't readily work with circuitkz
% Name with: \tikzsetnextfilename{<fig_file_name>}
%\tikzset{external/mode=graphics if exists} % Do not recompile if figure is present, without checking for changes. 
\NewDocumentEnvironment{dataplot}{mm}{\graphicspath{{Resources/Plots/#1/}} \input{Resources/Plots/#1/#2}}{}

\addbibresource{references.bib}

\begin{document}
\graphicspath{{Resources/Images/}{Resources/Diagrams/}}

\onehalfspacing
\pagenumbering{roman}

\title{Streamline plots for planar Filippov systems}
\author{Georgios A. Kafanas, Georgios Karagiannis}
\date{\today}

\maketitle

\begin{abstract}
A Python library to generate stream plots of planar Fillipov systems for Gnuplot scripts.
\end{abstract}

\thispagestyle{empty}
\restoregeometry

\clearpage

\tableofcontents
\listoffigures
\listoftables
\listofalgorithms

\clearpage

\pagenumbering{arabic}
%\begin{multicols}{2}

\section{Introduction}

The library transforms the output of the \texttt{streamplot} function of the \texttt{matplotlib.pyplot} Python plotting library in a form that can be plotted by Gnuplot. Smooth planar vector field and Filippov planar vector filed stream plots are supported. The input to the system is functions describing the vector field and the switching surface, and the output is a set of files in a folder with a name provided by the user. For a smooth planar field, there is an output file containing the stream lines, and a file containing the position of the direction arrows for the stream plot, positioned approximately in the middle of each streamline. For a planar Fillipov field, there is a pair for streamline and stream arrow files for each of the two smooth fields, and a file with the switching manifold.

\section{The plotting algorithm}

The \texttt{streamplot} function of the \texttt{matplotlib.pyplot} Python plotting library can be used to plot the stream plots of smooth planar fields. The output generated for a smooth planar field is a data structure that contains a list of unordered line sections, which when plotted create the stream-plots of a smooth planar field, and a set of graphical objects that describe the position of the direction arrows for the stream plots.

The position of the arrows cannot be easily extracted from the graphical objects. For this reason, the unordered line sections are collected in an ordered lists where each list corresponds to a streamline. A directional arrow is then placed approximately in the middle of each streamline. An additional advantage of the continuous ordered list representation of the streamline is that the list of point can be written directly in a Gnuplot input data file without repetition.

The planar Filippov system streamline plot package uses a hash map data structure to collect the line segments generate by the \texttt{streamplot} function of the \texttt{matplotlib.pyplot} Python plotting library in continuous lists. For instance a set of two stream line approximations
\begin{equation}
	S_{\mathrm{lines}} = \left\{ A_{0}B_{0}C_{0}D_{0}, A_{1}B_{1}C_{1}D_{1} \right\},
\end{equation}
where the points
\begin{equation}
	S_{\mathrm{points}} = \left\{ A_{0}, B_{0}, C_{0}, D_{0}, A_{1}, B_{1}, C_{1}, D_{1} \right\}
\end{equation}
are located at the edges of linear segments, may be provided by the \texttt{streamplot} algorithm as the following set of line segments
\begin{equation}
	S_{\mathrm{segments}} = \left\{ A_{0}B_{0}, B_{1} C_{1}, C_{0}D_{0}, A_{1}B_{1}, B_{0}C_{0}, C_{1} D_{1} \right\}.
\end{equation}

The algorithm which collect the line segments uses a map that references a list of continuous points from the floating point representation of the first and last point of the list. Four operations then applied repetitively to collect the segments in a contiguous series. Consider the list of points
\begin{equation}
	L_{\mathrm{segments}} = \left[ A_{0}B_{0}, B_{1} C_{1}, C_{0}D_{0}, A_{1}B_{1}, B_{0}C_{0}, C_{1} D_{1} \right].
\end{equation}
The following series of operations are applied to the hash table:
\begin{enumerate}
\item Start with an empty hash table:
	\begin{equation}
		H_{\mathrm{lines}} = \left\{ \right\}
	\end{equation}
\item \emph{Add} segment $A_{0} B_{0}$:
	\begin{equation}
		H_{\mathrm{lines}} = \left\{ (A_{0}, B_{0}) \to A_{0} B_{0} \right\}
	\end{equation}
\item \emph{Add} segment $B_{1} C_{1}$:
	\begin{equation}
		H_{\mathrm{lines}} = \left\{ (A_{0}, B_{0}) \to A_{0} B_{0}, (B_{1}, C_{1}) \to B_{1} C_{1} \right\}
	\end{equation}
\item \emph{Add} segment $C_{0} D_{0}$:
	\begin{equation}
		H_{\mathrm{lines}} = \left\{ (A_{0}, B_{0}) \to A_{0} B_{0}, (B_{1}, C_{1}) \to B_{1} C_{1}, (C_{0}, D_{0}) \to C_{0} D_{0} \right\}
	\end{equation}
\item \emph{Prepend} segment $A_{1} B_{1}$ to line $B_{1} C_{1}$:
	\begin{equation}
		H_{\mathrm{lines}} = \left\{ (A_{0}, B_{0}) \to A_{0} B_{0}, (A_{1}, C_{1}) \to A_{1} B_{1} C_{1}, (C_{0}, D_{0}) \to C_{0} D_{0} \right\}
	\end{equation}
\item \emph{Connect} segments $A_{0} B_{0}$ and $C_{0} D_{0}$ with segment $B_{0} C_{0}$:
	\begin{equation}
		H_{\mathrm{lines}} = \left\{ (A_{0}, D_{0}) \to A_{0} B_{0} C_{0} D_{0}, (A_{1}, C_{1}) \to A_{1} B_{1} C_{1} \right\}
	\end{equation}
\item \emph{Append} segment $C_{1} D_{1}$:
	\begin{equation}
		H_{\mathrm{lines}} = \left\{ (A_{0}, D_{0}) \to A_{0} B_{0} C_{0} D_{0}, (A_{1}, D_{1}) \to A_{1} B_{1} C_{1} D_{1} \right\}
	\end{equation}
\end{enumerate}
The four operations \emph{add}, \emph{prepend}, \emph{append}, and \emph{connect} are applied according to the edges of the line segments that are available in each step in the map $H_{\mathrm{lines}}$.

The arrows along the streamlines are provided by \texttt{streamplot} as a set of graphics objects which do not maintain any direct record of coordinates with their position. Thus, a new arrow is generated and placed in the middle of each stream plot line. The body of the arrow is a line segment whose middle coincides with the middle of the stream plot line, and extends up to the closest edge of the line segment where the middle of the stream line is located, in a symmetric manner. The extend of the direction arrow is bounded below by a given percentage of the length of the segment that contains  the middle of the stream line to avoid numerical instabilities. Thus, the arrow may extend beyond the segment, but its extend will be sufficiently small to be hidden by the arrow figure.

\phantomsection % Command of the hyperref package
\addcontentsline{toc}{section}{Bibliography}
\printbibliography[title={References}] % heading=subbibliography
%\bibliography{references}
%\bibliographystyle{ieeetr}

%\appendix
%\appendixpage
%\addappheadtotoc

%\section{Introduction} \label{A:Introduction}

%smth.
%\lstinputlisting[language=Matlab, breaklines=true, morekeywords={matlab2tikz}, basicstyle=\small\ttfamily, commentstyle=\usefont{T1}{pcr}{m}{sl}\small]{Source/code.m}

%\end{multicols}
\end{document}
