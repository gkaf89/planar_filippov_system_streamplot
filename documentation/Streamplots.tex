\documentclass[english, a4paper]{article}

\usepackage{StyleFile}
\usepackage{LayoutFile}

\newcommand{\makeDiagramPath}[1]{Resources/Diagrams/#1}
\tikzexternalize[prefix=tikz-gen/] % Note: doesn't readily work with circuitkz
% Name with: \tikzsetnextfilename{<fig_file_name>}
%\tikzset{external/mode=graphics if exists} % Do not recompile if figure is present, without checking for changes. 
\NewDocumentEnvironment{dataplot}{mm}{\graphicspath{{Resources/Plots/#1/}} \input{Resources/Plots/#1/#2}}{}

\addbibresource{references.bib}

\begin{document}
\graphicspath{{Resources/Images/}{Resources/Diagrams/}}

\onehalfspacing
\pagenumbering{roman}

\title{Stream plots}
\author{Georgios A. Kafanas, Georgios Karagiannis}
\date{\today}

\maketitle

\begin{abstract}
A Python library to generate stream plots of planar Fillipov systems for Gnuplot scripts.
\end{abstract}

\thispagestyle{empty}
\restoregeometry

\clearpage

\tableofcontents
\listoffigures
\listoftables
\listofalgorithms

\clearpage

\pagenumbering{arabic}
%\begin{multicols}{2}

\section{Introduction}

The library uses a hash map to connect the output of Python piloting libraries into continuous line and arrow data files that Gnuplot can then plot. The function \texttt{streamplot} of the \texttt{matplotlib.pyplot} package, provides a list of line sections in an unordered set. To plot the lines with Gnuplot, the line segments must be placed in order along their corresponding lines, and the lines must be written in an unordered manner as a series of points.

For instance a set of two stream line approximations
\begin{equation}
	S_{\mathrm{lines}} = \left\{ A_{0}B_{0}C_{0}D_{0}, A_{1}B_{1}C_{1}D_{1} \right\},
\end{equation}
where the points
\begin{equation}
	S_{\mathrm{points}} = \left\{ A_{0}, B_{0}, C_{0}, D_{0}, A_{1}, B_{1}, C_{1}, D_{1} \right\}
\end{equation}
are located at the edges of linear segments, may be provided by the \texttt{streamplot} algorithm as the following set of line segments
\begin{equation}
	S_{\mathrm{segments}} = \left\{ A_{0}B_{0}, B_{1} C_{1}, C_{0}D_{0}, A_{1}B_{1}, B_{0}C_{0}, C_{1} D_{1} \right\}.
\end{equation}

A hash table is used to collect the line segments in contiguous manner. The hash table links each contiguous series of points in a line, to a key with the first and last point of the series. Four operations are implemented to collect the segments in a contiguous series. Consider the list of points
\begin{equation}
	L_{\mathrm{segments}} = \left[ A_{0}B_{0}, B_{1} C_{1}, C_{0}D_{0}, A_{1}B_{1}, B_{0}C_{0}, C_{1} D_{1} \right].
\end{equation}
The following series of operations are applied to the hash table:
\begin{enumerate}
\item Start with an empty hash table:
	\begin{equation}
		H_{\mathrm{lines}} = \left\{ \right\}
	\end{equation}
\item \emph{Add} segment $A_{0} B_{0}$:
	\begin{equation}
		H_{\mathrm{lines}} = \left\{ (A_{0}, B_{0}) \to A_{0} B_{0} \right\}
	\end{equation}
\item \emph{Add} segment $B_{1} C_{1}$:
	\begin{equation}
		H_{\mathrm{lines}} = \left\{ (A_{0}, B_{0}) \to A_{0} B_{0}, (B_{1}, C_{1}) \to B_{1} C_{1} \right\}
	\end{equation}
\item \emph{Add} segment $C_{0} D_{0}$:
	\begin{equation}
		H_{\mathrm{lines}} = \left\{ (A_{0}, B_{0}) \to A_{0} B_{0}, (B_{1}, C_{1}) \to B_{1} C_{1}, (C_{0}, D_{0}) \to C_{0} D_{0} \right\}
	\end{equation}
\item \emph{Prepend} segment $A_{1} B_{1}$ to line $B_{1} C_{1}$:
	\begin{equation}
		H_{\mathrm{lines}} = \left\{ (A_{0}, B_{0}) \to A_{0} B_{0}, (A_{1}, C_{1}) \to A_{1} B_{1} C_{1}, (C_{0}, D_{0}) \to C_{0} D_{0} \right\}
	\end{equation}
\item \emph{Connect} segments $A_{0} B_{0}$ and $C_{0} D_{0}$ with segment $B_{0} C_{0}$:
	\begin{equation}
		H_{\mathrm{lines}} = \left\{ (A_{0}, D_{0}) \to A_{0} B_{0} C_{0} D_{0}, (A_{1}, C_{1}) \to A_{1} B_{1} C_{1} \right\}
	\end{equation}
\item \emph{Append} segment $C_{1} D_{1}$:
	\begin{equation}
		H_{\mathrm{lines}} = \left\{ (A_{0}, D_{0}) \to A_{0} B_{0} C_{0} D_{0}, (A_{1}, D_{1}) \to A_{1} B_{1} C_{1} D_{1} \right\}
	\end{equation}
\end{enumerate}
Thus, the four operations \emph{add}, \emph{prepend}, \emph{append}, and \emph{connect} are implemented to connect the segments that are generated by \texttt{streamplot}.

The arrows along the streamlines are provided by \texttt{streamplot} as a set of graphics objects which do not maintain any direct record of coordinates with their position. Thus, a new arrow is generated and placed in the middle of each stream plot line. The body of the arrow is a line segment whose middle coincides with the middle of the stream plot line, and extends up to the closest edge of the line segment where the middle of the stream line is located, in a symmetric manner. The extend of the direction arrow is bounded below by a given percentage of the length of the segment that contains  the middle of the stream line to avoid numerical instabilities. Thus, the arrow may extend beyond the segment, but its extend will be sufficiently small to be hidden by the arrow figure.

\phantomsection % Command of the hyperref package
\addcontentsline{toc}{section}{Bibliography}
\printbibliography[title={References}] % heading=subbibliography
%\bibliography{references}
%\bibliographystyle{ieeetr}

%\appendix
%\appendixpage
%\addappheadtotoc

%\section{Introduction} \label{A:Introduction}

%smth.
%\lstinputlisting[language=Matlab, breaklines=true, morekeywords={matlab2tikz}, basicstyle=\small\ttfamily, commentstyle=\usefont{T1}{pcr}{m}{sl}\small]{Source/code.m}

%\end{multicols}
\end{document}
